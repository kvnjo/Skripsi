\chapter{Analisa Masalah}
\label{chap:analisa}

	Pada bab ini akan dibahas mengenai hasil analisa permasalahan flow shop dan segalah hal yang berkaitan dengan analisa permasalahan tersebut.
	Pada bab ini juga akan dibahas sekilas mengenai rancangan awal dari perangkat lunak yang memungkinkan diaplikasikannya
	algoritma ant colony pada permasalahan multi objective flow shop.
	
	
\section{Analisa Kasus}

	Proses penjadwalan Flow Shop adalah salah satu metode penjadwalan produksi di mana urutan mesin 
	yang digunakan untuk setiap proses dalam seluruh pekerjaan harus sama. Pekerjaan-pekerjaan tersebut
	akan dikerjakan pada mesin-mesin yang ada dengan urutan pengerjaan yang dipilih. Masing - masing
	proses akan dikerjakan secara sistematis dan terurut. Data-data masukan yang dibutuhkan
	dalam menjalankan sebuah proses penjadwalan flow shop adalah:
	\begin{itemize}
		\item Banyaknya pekerjaan
		\item Banyaknya mesin
		\item Detail waktu pengerjaan setiap pekerjaan
		
	\end{itemize}

	Setiap pekerjaan memiliki waktu proses yang berbeda satu sama lain pada tiap mesin. Variasi
	urutan pengerjaan yang berbeda biasanya akan menghasilkan waktu pengerjaan / makespan yang
	berbeda pula. Proses penjadwalan ini penting untuk diperhatikan, karena mampu mempengaruhi
	lama waktu penggunaan fasilitas dalam suatu proses produksi. Urutan pengerjaan yang baik dapat
	dicari dengan menggunakan suatu algoritma optimisasi. Diharapkan dengan algoritma optimisasi
	tersebut, urutan pengerjaan yang menghasilkan makespan / waktu pengerjaan minimum dapat
	ditemukan.
	
	Salah satu algoritma yang dapat digunakan untuk optimisasi penjadwalan flow shop
	adalah algoritma ant colony. Tingkat keoptimalan setiap algoritma berbeda-beda, oleh karena
	itu perlu dibuat sebuah perangkat lunak untuk optimisasi penjadwalan flow shop dengan
	menggunakan algoritma ant colony. Algoritma ant colony akan menerima data masukan dari suatu
	kasus flow shop, kemudian akan mencari urutan pengerjaan / solusi yang paling optimal dari
	kasus tersebut selain itu kita juga dapat mencari kriteria lain yang kita inginkan.
	
	Algoritma ant colony akan mencari urutan pengerjaan yang menghasilkan makespan / waktu
	pengerjaan yang paling minimum. Algoritma ant colony merepresentasikan pilihan solusi-solusi
	yang ada sebagai jalur yang akan dilalui oleh semut dan memberikan data feromon sebagai panduan
	untuk melewati jalur-jalur tersebut. Dengan panduan tersebut, semut-semut tersebut diasumsikan
	mampu memilih jalur yang paling optimal.
	Proses optimisasi dengan menggunakan algoritma ant colony perlu memperhatikan beberapa
	hal. 
	
	Algoritma ant colony perlu mengetahui bagaimana cara merepresentasikan suatu solusi sebagai
	jalur, bagaimana cara menggunakan panduan feromon untuk pemilihan jalur, bagaimana cara
	membaca dan menyimpan suatu data feromon, serta banyak hal lainnya. Tata cara tersebut mampu
	mempengaruhi hasil optimisasi dari algoritma ant colony.
\section{Analisa Algoritma Ant Colony Pada Penjadwalan Flow Shop}
\subsection{Urutan Pengerjaan Proses Sebagai Jalur Semut}
Algoritma ant colony akan membantu penentuan urutan pengambilan pekerjaan yang optimal. Oleh
karena itu, algoritma ini akan merepresentasikan pilihan urutan pengambilan yang ada sebagai jalur. Jalur-jalur tersebut kemudian akan dilewati oleh semut-semut yang akan mencari solusi / urutan pengerjaan yang optimal. Pilihan urutan pengerjaan yang ada dapat dicari dengan menggunakan fungsi permutasi terhadap jumlah pekerjaan.

Sebagai contoh, jika dimisalkan terdapat 3 buah pekerjaan, maka dengan fungsi permutasi akan
didapatkan jalur-jalur urutan pengerjaan (1,2,3), (1,3,2), (2,1,3), (2,3,1), (3,1,2), dan (3,2,1). Fungsi
permutasi ini digunakan dengan mempertimbangkan sifat dari proses penjadwalan flow shop.
Masing-masing proses dari pekerjaan pasti akan dikerjakan dan hanya akan dikerjakan sebanyak
satu kali.

\subsection{Memilih Urutan Pengerjaan / Jalur}
Makespan yang dihasilkan oleh masing-masing urutan pengerjaan biasanya akan bernilai berbeda.
Penentuan urutan pengerjaan yang optimal akan ditentukan oleh nilai feromon yang akan diberikan
oleh semut-semut yang telah disebar. Nilai feromon tersebut dapat disimpan dengan cara yang
beragam.

Dalam kasus hybrid ow shop juga terdapat beragam cara dalam menyimpan feromon dan
penentuan urutan yang lebih optimal. Salah satunya adalah dengan menyimpan nilai feromon
sebagai panduan pemilihan pekerjaan selanjutnya. Nilai feromon akan menyimpan kecenderungan
dipilihnya suatu pekerjaan setelah dipilihnya suatu pekerjaan yang lain. Nilai feromon tersebut
akan disimpan dalam bentuk matriks 2 dimensi.

Indeks pertama dari matriks akan merepresentasikan nomor pekerjaan yang sebelumnya dipilih.
Indeks kedua akan merepresentasikan pekerjaan yang selanjutnya akan dipilih. Sebagai contoh,
pada matriks indeks [1][2] menunjukkan kecenderungan dipilihnya pekerjaan 2 setelah dipilihnya
pekerjaan 1. Nilai feromon pada matriks dengan 2 indeks yang sama akan selalu bernilai 0.

Perlu diingat pula bahwa algoritma ant colony akan selalu memberikan peluang dipilihnya sebuah
solusi. Meskipun solusi tersebut sudah dapat dianggap sebagai solusi yang kurang optimal,
solusi tersebut harus tetap memiliki kemungkinan untuk dipilih. Oleh karena itu, indeks nilai feromon
harus tetap dapat mengacu pada terpilihnya suatu solusi-solusi yang ada. Nilai feromon
masing-masing indeks matriks (kecuali matriks dengan dua indeks yang sama) tidak akan lebih kecil
dari 1. Hal ini dilakukan agar setiap solusi masih memiliki kemungkinan untuk dipilih, meskipun
kemungkinannya sangat kecil.

\subsection{Proses Pemilihan Urutan Pengerjaan / Jalur Semut}
Pada proses pemilihan jalur, nilai feromon yang disimpan oleh suatu indeks matriks mempengaruhi
kemungkinan dipilihnya suatu pekerjaan. Semakin besar nilai feromon pada suatu indeks, semakin
besar kemungkinan dipilihnya pekerjaan yang dirujuk oleh indeks matriks tersebut. Berdasarkan
rumus untuk pemilihan jalur pada algoritma ant colony, diperlukan nilai objektif dan nilai feromon
untuk penentuan pemilihan suatu jalur. Pada proses pemilihan pekerjaan ini, nilai objektif
yang digunakan adalah 1 untuk semua indeks matriks. Proses pemilihan pekerjaan ini hanya akan
dipengaruhi oleh nilai feromon yang disimpan pada indeks matriks yang ada.

Penentuan pekerjaan mana yang akan dipilih dapat dimulai dengan melakukan proses randomisasi
suatu angka. Angka tersebut akan berada di kisaran nilai 1 hingga jumlah nilai feromon
dari indeks-indeks yang mungkin dilibatkan. Penentuan pekerjaan dilakukan dengan menambahkan
secara satu per satu nilai feromon dari indeks yang terlibat dimulai dari nilai 0. Jika setelah
ditambahkan nilai dari suatu indeks, jumlah nilai telah melebihi angka hasil randomisasi, maka
pekerjaan yang dirujuk indeks tersebut akan dipilih sebagai pekerjaan yang diambil.

\subsection{Rumus - rumus ACO dalam penjadwalan Flow Shop}
\begin{enumerate}
	\item Pemilihan Jalur\\
	Kemungkinan dipilihnya sebuah jalur dipengaruhi oleh kekuatan feromon pada jalur tersebut. Semakin
	kuat / banyak feromon pada suatu jalur, maka jalur tersebut akan lebih sering / berkemungkinan
	lebih besar untuk dipilih. Kemungkinan terpilihnya masing-masing jalur dapat dihitung
	dengan menggunakan rumus yang terdapat pada rumus berikut :
	
	\begin{equation}
	P_{k}^{xy} = \frac{T_{xy}^{\alpha}\cdot n_{xy}^{\beta}}{\sum_{x \in availeble_z}(T_{xz}^{\alpha }\cdot n_{xz}^{\beta })}
	\end{equation}\\
	
	
	Keterangan dari rumus tersebut sebgai berikut :
	
	$P_{k}^{xy}$ : nilai kemungkinan terpilihnya jalur xy oleh semut k \\
	$T_{xy}$ : nilai feromon untuk jalur xy\\
	$n_{xy}$ : nilai objektif untuk jalur xy\\
	$\alpha$ : persentase pengaruh nilai feromon\\
	$\beta$  : persentase pengaruh nilai objektif\\
	
	
	T menunjukkan kekuatan / jumlah feromon yang terdapat pada suatu jalur, sedangkan n menunjukkan
	nilai objektif yang dimiliki oleh suatu jalur. Nilai objektif pada suatu jalur menunjukkan
	pengetahuan / nilai awal mengenai jalur tersebut. Nilai awal tersebut biasanya berupa informasi
	mengenai jalur tersebut yang diketahui sejak awal dan mampu memberikan pengaruh pada proses
	pemilihan jalur.
	
	Contoh informasi yang dapat menjadi nilai objektif adalah panjang jalur, banyaknya belokan,
	panjang jalur pertama, dan lain-lain. Nilai awal suatu jalur dapat bernilai sama dengan jalur yang
	lain. Apabila proses pencarian yang dilakukan tidak memiliki informasi apapun atau bersifat blind-
	search, nilai objektif ini dapat diabaikan. Besar pengaruh dari kekuatan feromon dan nilai objektif
	terhadap kemungkinan terpilihnya sebuah jalur dapat diatur dengan menggunakan rasio tertentu.
	
	\item Update Nilai Feromon\\
	Ketika terdapat semut yang menyelesaikan sebuah jalur, feromon pada jalur tersebut akan diubah.
	Jumlah feromon pada jalur yang dilewati semut tersebut akan ditambah. Dalam proses penambahan
	feromon, nilai feromon yang baru pada jalur ditentukan dengan rumus berikut:
	
	\begin{equation}
	T_{xy} \leftarrow (1-P) \cdot T_{xy} + \sum_{k}\Delta T_{xy}^k
	\end{equation}\\
	
	Keterangan dari rumus tersebut sebgai berikut :
	
	$P$ : persentase evaporasi feromon (dalam desimal)\\
	$T_{xy}$ : nilai feromon untuk jalur xy\\
	$\Delta T_{xy}^k$ : nilai feromon yang akan ditambahkan oleh semut k pada jalur xy\\
	
	Pada rumus tersebut, jumlah feromon awal yang terdapat pada suatu jalur akan mengalami
	pengurangan berdasarkan rasio penguapan feromon P, lalu kemudian akan ditambah dengan nilai
	feromon masing-masing semut yang melewati jalur tersebut. Nilai feromon masing-masing semut
	dapat berupa sebuah nilai tetap atau nilai lainnya yang mampu memberikan nilai lebih pada jalur
	yang lebih cepat (lama proses, jumlah semut yang telah melewati suatu jalur, dan lain-lain).
	Aturan mengenai kapan dan bagaimana suatu proses penambahan feromon dilakukan dapat
	bervariasi. Aturan-aturan tersebut biasanya ditentukan berdasarkan jenis permasalahan yang ingin
	dioptimisasi. Proses penambahan feromon dapat dilakukan ketika terdapat semut yang telah selesai
	berjalan pada jalurnya atau pada saat seluruh semut telah selesai berjalan pada jalurnya. 
	
	Proses perubahan jumlah feromon dapat dilakukan dengan menambahkan sebuah nilai atau penambahkan
	nilai feromon dengan suatu persentase tertentu.
	Proses perubahan jumlah feromon dapat dilakukan di seluruh jalur atau hanya di jalur yang
	dilewati suatu semut. Jika perubahan dilakukan di seluruh jalur, fungsi / aturan khusus untuk
	perubahan jumlah feromon sebagai berikut.
	
	\begin{equation}
	\Delta T_{xy}^k = \begin{cases}
	Q, \mbox{jika semut {\bf k} menggunakan jalur {\bf xy}} .\newline
	0, \mbox{untuk lainnya}.
	\end{cases}
	\end{equation}\\
	
	
	Keterangam aturan khusus di atas sebagai berikut :
	
	$\Delta T_{xy}^k$ : nilai feromon yang akan ditambahkan oleh semut k pada jalur xy\\
	$Q$ 				: nilai penambahan feromon (dapat berupa suatu angka tetap / hasil perhitungan)\\
	
	\item Kondisi berhenti\\
	Selama proses optimisasi, algoritma ant colony akan secara terus menerus melakukan fase pelatihan.
	Pada setiap fase pelatihan akan dibentuk semut-semut yang akan memilih suatu jalur secara acak
	berdasarkan nilai feromon. Setelah suatu fase pelatihan selesai, perubahan nilai feromon akan
	dilakukan berdasarkan jalur-jalur yang dipilih oleh masing-masing semut.
	
	Proses optimisasi akan diberhentikan jika suatu kondisi telah terpenuhi. Waktu berhentinya
	suatu proses pelatihan dapat ditentukan dengan berbagai parameter. Proses pelatihan dapat dianggap
	selesai jika telah melewati beberapa fase pelatihan, jumlah semut yang disebar sudah cukup
	banyak, atau jika sudah terdapat jalur yang dipilih oleh sebagian besar semut. Proses pelatihan
	juga dapat diberhentikan jika fase-fase pelatihan terakhir selalu memberikan hasil solusi yang sama
	/ tidak mampu membentuk solusi lain yang lebih optimal.
\end{enumerate}
