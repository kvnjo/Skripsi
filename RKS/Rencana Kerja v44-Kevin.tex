\documentclass[a4paper,twoside]{article}
\usepackage[T1]{fontenc}
\usepackage[bahasa]{babel}
\usepackage{graphicx}
\usepackage{graphics}
\usepackage{enumitem}
\usepackage{float}
\usepackage[cm]{fullpage}
\pagestyle{myheadings}
\usepackage{etoolbox}
\usepackage{setspace} 
\usepackage{lipsum} 
\setlength{\headsep}{30pt}
\usepackage[inner=2cm,outer=2.5cm,top=2.5cm,bottom=2cm]{geometry} %margin
% \pagestyle{empty}

\makeatletter
\renewcommand{\@maketitle} {\begin{center} {\LARGE \textbf{ \textsc{\@title}} \par} \bigskip {\large \textbf{\textsc{\@author}} }\end{center} }
\renewcommand{\thispagestyle}[1]{}
\markright{\textbf{\textsc{AIF401/AIF402 \textemdash Rencana Kerja Skripsi \textemdash Sem. Genap 2016/2017}}}

\newcommand{\HRule}{\rule{\linewidth}{0.4mm}}
\renewcommand{\baselinestretch}{1}
\setlength{\parindent}{0 pt}
\setlength{\parskip}{6 pt}

\onehalfspacing
 
\begin{document}

\title{\@judultopik}
\author{\nama \textendash \@npm} 

%tulis nama dan NPM anda di sini:
\newcommand{\nama}{Kevin Jonathan}
\newcommand{\@npm}{2014730020}
\newcommand{\@judultopik}{Ant Colony Optimization (ACO) untuk Permasalahan Multi Objective Flowshop Scheduling (MOFSP)} % Judul/topik anda
\newcommand{\jumpemb}{1} % Jumlah pembimbing, 1 atau 2
\newcommand{\tanggal}{14/02/2018}

% Dokumen hasil template ini harus dicetak bolak-balik !!!!
%\renewcommand{\labelenumi}{\alph{enumi}.}

\maketitle

\pagenumbering{arabic}

\section{Deskripsi}
%Tuliskan deskripsi dari topik skripsi yang akan anda ajukan. Di sini dapat dituliskan latar belakang, seperti apa penelitian yang sudah ada sebelumnya dan apa yang akan anda kerjakan. Sertakan gambar agar penjelasan anda menjadi lebih baik.

%Pada skripsi ini, akan dibuat sebuah perangkat lunak yang dapat menampilkan visualisasi dan simulasi kerumunan orang yang berkunjung ke sebuah museum. Dengan menggunakan perangkat lunak tersebut, pengelola museum dapat mengatur tempat peletakan objek sehingga tidak terjadi kerumunan yang terlalu padat.

%Dari berbagai macam teknik yang dapat digunakan untuk melakukan simulasi kerumunan, dipilih dua buah teknik yaitu teknik {\it flow tiles} dan {\it social force model (steering behaviour)}.

%Dst, dst, dst, \ldots\ldots\ldots 

%Perangkat lunak akan dibuat dengan bantuan {\it framework} OpenSteer. Sebagai studi kasus, museum yang digunakan untuk melakukan simulasi adalah Museum Geologi Bandung.

%Dst, dst, dst, \ldots\ldots\ldots 

%flowshop scheduling problem........
%Jelasin jenis penjadwalan pada proses %produksi.......
Penjadwalan produksi merupakan aktivitas yang tidak terpisahkan dalam suatu perusahaan {\it manufakturing}. Penjadwalan ({\it scheduling}) sendiri didefinisikan sebagai suatu proses pengalokasian sumber daya atau mesin-mesin yang ada untuk melaksanakan tugas-tugas yang ada dalam suatu waktu tertentu (Baker, 1974). Sedangkan yang dimaksud dengan proses produksi adalah serangkaian langkah-langkah yang digunakan untuk mengtransformasi {\it Input} menjadi {\it Output}.

Proses penjadwalan {\it Flow Shop} adalah salah satu metode penjadwalan produksi di mana urutan mesin yang digunakan untuk setiap proses dalam seluruh pekerjaan harus sama. Dalam penelitian - penelitian penjadwalan sebelumnya hanya difokuskan pada satu kriteria saja ({\it single}) namun pada penelitian kali ini akan menggunakan lebih dari satu kriteria ({\it multiple}). Banyak algoritma yang dapat digunakan untuk menentukan urutan  pengerjaan pekerjaan dalam proses penjadwalan produksi {\it Flow Shop}. Salah satu algoritma yang dapat digunakan dalam proses penjadwalan produksi {\it Multi Objective Flow Shop} adalah algoritma {\it Ant Colony Optimization}. Algoritma {\it Ant Colony Optimization} adalah algoritma yang mengadopsi perilaku koloni semut yang dikenal sebagai sistem semut. Algoritma ini menyelesaikan permasalahan berdasarkan tingkah laku semut dalam sebuah koloni yang sedang mencari sumber makanan.

Penelitian ini dibuat untuk mempelajari, mengaplikasikan, serta mengukur kinerja Algoritma {\it Ant Colony Optimization} pada proses penjadwalan {\it Multi Objective Flow Shop Scheduling} (MOFSP). Pada skripsi ini juga akan dibuat perangkat lunak yang dapat menerima n job yang masing-masing terdiri atas m buah operasi dan m buah mesin. Setiap operasi hanya ditangani oleh sebuah mesin dan setiap mesin hanya bisa menangani satu operasi. Urutan operasi dari setiap job adalah sama.


\section{Rumusan Masalah}
%Tuliskan rumusan dari masalah yang akan anda bahas pada skripsi ini. Rumusan masalah biasanya berupa kalimat pertanyaan. Gunakan itemize seperti contoh di bagian Deskripsi Perangkat Lunak.
\begin{enumerate}[label=(\alph*)]
	\item Apa itu penjadwalan {\it Multi Objective Flowshop Scheduling} (MOFSP) ?
	\item Bagaimana mengimplementasikan algoritma {\it Ant Colony Optimization} (ACO) untuk menyelesaikan permasalahan MOFSP ?
	\item Bagaimana hasil penyelesaian masalah MOFSP dengan menggunakan algoritma {\it Ant Colony Optimization} (ACO) ?
	\item Bagaimana kelebihan dan kekurangan penyelesaian masalah MOFSP dengan menggunakan algoritma {\it Ant Colony Optimization} (ACO) ?
	\item Seberapa maksimal algoritma {\it Ant Colony Optimization} (ACO) dalam menyelesaikan permasalahan MOFSP ?

\end{enumerate}

\section{Tujuan}
%Tuliskan tujuan dari topik skripsi yang anda ajukan. Tujuan penelitian biasanya berkaitan erat dengan pertanyaan yang diajukan di bagian rumusan masalah. Gunakan itemize seperti contoh di bagian Deskripsi Perangkat Lunak.
\begin{enumerate}[label=(\alph*)]
	\item Menjelaskan penjadwalan {\it Multi Objective Flowshop Scheduling} (MOFSP).
	\item Mengimplementasikan algoritma {\it Ant Colony Optimization} (ACO) untuk menyelesaikan permasalahan MOFSP.
	\item Menampilkan hasil penyelesaian masalah MOFSP dengan menggunakan algoritma {\it Ant Colony Optimization} (ACO).
	\item Menampilkan kelebihan dan kekurangan penyelesaian masalah MOFSP dengan menggunakan algoritma {\it Ant Colony Optimization} (ACO) ?
	\item Mengetahui seberapa maksimal algoritma {\it Ant Colony Optimization} (ACO) dalam menyelesaikan permasalahan MOFSP.
	
\end{enumerate}


\section{Deskripsi Perangkat Lunak}
%Tuliskan deksripsi dari perangkat lunak yang akan anda hasilkan. Apa saja fitur yang disediakan oleh PL tersebut dan apa saja kemampuan dari PL tersebut. Perhatikan contoh di bawah ini:
Perangkat lunak akhir yang akan dibuat memiliki fitur minimal sebagai berikut:
\begin{itemize}
	\item Perangkat lunak dapat menerima kasus MOFSP untuk dicari solusinya.
	\item Perangkat lunak dapat menjalankan algoritma {\it Ant Colony Optimization}.
	\item Perangkat lunak dapat menghasilkan proses optimasi menggunakan algoritma {\it Ant Colony Optimization}.
	\item Perangkat lunak dapat menghasilakan urutan pengerjaan atau solusi terbaik untuk permasalahan MOFSP.
	
\end{itemize}

\section{Detail Pengerjaan Skripsi}
%Tuliskan bagian-bagian pengerjaan skripsi secara detail. Bagian pekerjaan tersebut mencakup awal hingga akhir skripsi, termasuk di dalamnya pengerjaan dokumentasi skripsi, pengujian, survei, dll.
Bagian-bagian pekerjaan skripsi ini adalah sebagai berikut :
	\begin{enumerate}
		\item Melakukan studi literatur : penjadwalan proses produksi secara umum, MOFSP, ACO, dan aplikasi ACO untuk masalah penjadwalan.
		\item Melakukan analisa aplikasi ACO pada masalah MOFSFP.
		\item Mengembangkan perangkat lunak (analisis, desain, implementasi, dan pengujian).
		\item Melakukan eksperimen dengan sebuah {\it benchmark}. 
		\item Menulis dokumen skripsi.
	\end{enumerate}

\section{Rencana Kerja}
Rincian capaian yang direncanakan di Skripsi 1 adalah sebagai berikut:
\begin{enumerate}
\item Melakukan studi literatur penjadwalan proses produksi secara umum.
\item Melakukan studi literatur penjadwalan {\it Multi Objective Flowshop Scheduling} (MOFSP).
\item Melakukan studi literatur ACO serta pengaplikasiannya pada permasalahan MOFSP.
\item Menulis sebagian dokumen skripsi, dari bab 1 sampai dengan sebagian bab 2.

\end{enumerate}

Sedangkan yang akan diselesaikan di Skripsi 2 adalah sebagai berikut:
\begin{enumerate}
\item Merancang perangkat lunak.
\item Menyelesaikan pembangunan perangkat lunak.
\item Melakukan eksperimen dengan sebuah {\it benchmark}.
\item Menyelesaikan penulisan dokumen skripsi.

\end{enumerate}

\newpage
\vspace{1cm}
\centering Bandung, \tanggal\\
\vspace{2cm} \nama \\ 
\vspace{1cm}

Menyetujui, \\
\ifdefstring{\jumpemb}{2}{
\vspace{1.5cm}
\begin{centering} Menyetujui,\\ \end{centering} \vspace{0.75cm}
\begin{minipage}[b]{0.45\linewidth}
% \centering Bandung, \makebox[0.5cm]{\hrulefill}/\makebox[0.5cm]{\hrulefill}/2013 \\
\vspace{2cm} Nama: \makebox[3cm]{\hrulefill}\\ Pembimbing Utama
\end{minipage} \hspace{0.5cm}
\begin{minipage}[b]{0.45\linewidth}
% \centering Bandung, \makebox[0.5cm]{\hrulefill}/\makebox[0.5cm]{\hrulefill}/2013\\
\vspace{2cm} Nama: \makebox[3cm]{\hrulefill}\\ Pembimbing Pendamping
\end{minipage}
\vspace{0.5cm}
}{
% \centering Bandung, \makebox[0.5cm]{\hrulefill}/\makebox[0.5cm]{\hrulefill}/2013\\
\vspace{2cm} Nama: \makebox[3cm]{\hrulefill}\\ Pembimbing Tunggal
}
\end{document}

