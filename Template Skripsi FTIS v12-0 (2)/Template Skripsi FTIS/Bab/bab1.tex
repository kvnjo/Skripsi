%versi 2 (8-10-2016) 

\chapter{Pendahuluan}
\label{chap:intro}
   
\section{Latar Belakang}
\label{sec:label}

%Bagian ini akan diisi dengan apa yang melatarbelakangi pembuatan template skripsi ini.
%Termasuk juga masalah-masalah yang akan dihadapi untuk membuatnya, termasuk kurangnya kemampuan penguasaan \LaTeX{} sehingga template ini dibuat dengan mengandalkan berbagai contoh yang tersebar di dunia maya, yang digabung-gabung menjadi satu jua.
%Bagian lain juga akan dilengkapi, untuk sementara diisi dengan lorem ipsum versi bahasa inggris.

Data graf dapat berukuran sangat besar. Salah satu contoh representasi dari graf yang berukuran besar adalah jaringan relasi pertemanan pada media sosial. Misalnya pada media sosial {\it Facebook}, dimana simpulnya adalah para pengguna {\it Facebook} sedangkan sisi-sisinya adalah pertemanan antar penggunanya.

Agar dapat dianalisis pada sistem pararel, graf tersebut perlu dipecah-pecah menjadi sub-graf lalu disimpan pada {\it node-node} {\it slave} dari masing-masing komputer. Hasil analisis dari pemecahan graf dapat digunakan untuk kepentingan yang lebih lanjut. Pemecahan graf tersebut dapat menggunakan bantuan berbagai macam algoritma pemecahan graf yang sudah ada.

Pada skripsi ini, akan dibuat perangkat lunak yang dapat menerima {\it input} data graf dan mengolahnya sedemikian rupa dengan struktur data yang tepat sehingga menghasilkan {\it output} berupa sub-graf yang beraturan dan seimbang. Perangkat lunak ini juga akan menampilkan hasil visualisasi dari {\it input} dan {\it output} pemecahan graf. Dari berbagai macam algoritma yang dapat digunakan untuk memecah graf, dipilih satu algoritma yaitu algoritma {\it vertex cut}.


%\dtext{5-10}

\section{Rumusan Masalah}
\label{sec:rumusan}
%Bagian ini akan diisi dengan penajaman dari masalah-masalah yang sudah diidentifikasi di bagian sebelumnya. 
\begin{itemize}
	\item Bagaimana merancang struktur graf ke dalam struktur data perangkat lunak ?
	\item Bagaimana cara memvisualisasikan graf dengan benar ?
	\item Bagaimana cara merumuskan algoritma yang tepat untuk melakukan pemecahan graf berbasis {\it vertex}?
	\item Bagaimana membangun perangakat lunak yang dapat memecah graf berbasis {\it vertex} dari berbagai data graf ?
	\item Bagaimana perbandingan hasil pemecahan graf dengan pemecahan graf menggunakan sistem terdistribusi : {\it GraphX} pada {\it Spark} ?

\end{itemize}


%\dtext{6}

\section{Tujuan}
\label{sec:tujuan}
Berdasarkan rumusan masalah tersebut, tujuan dari tugas akhir ini adalah sebagai berikut :
\begin{itemize}
\item Melakukan perancangan data graf ke dalam stuktur data yang tepat. 
\item Menampilkan visualisasi graf dalam format grafis.
\item Mempelajari dan mengimplementasikan teknik {\it vertex cut}.
\item Merancang dan membangun perangkat lunak berbasis {\it vertex} yang dapat memecah graf dari berbagai data graf.
\item Membandingkan hasil pemecahan graf dengan pemecahan graf menggunakan sistem terdistribusi : {\it GraphX} pada {\it Spark}.
\end{itemize}


\section{Batasan Masalah}
\label{sec:batasan}
%Untuk mempermudah pembuatan template ini, tentu ada hal-hal yang harus dibatasi, misalnya saja bahwa template ini bukan berupa style \LaTeX{} pada umumnya (dengan alasannya karena belum mampu jika diminta membuat seperti itu)

%\dtext{8}

\section{Metodologi}
\label{sec:metlit}
%Tentunya akan diisi dengan metodologi yang serius sehingga templatenya terkesan lebih serius.

%\dtext{9}

Penyusunan tugas akhir ini menggunakan metodologi sebagai berikut :
\begin{enumerate}
	\item Melakukan studi literatur : graf dan pemecahan graf untuk lingkungan pararel.
		\item Melakukan studi literatur algoritma pemecahan graf.
		\item Mengolah struktur data graf ke dalam stuktur data perangkat lunak.
		\item Merancang visualisasi graf.
		\item Merumuskan algoritma pemecahan graf yang tepat dan sesuai.
		\item Mengimplementasikan teknik {\it vertex cut} untuk melakukan pemecahan graf.
		\item Melakukan studi literatur : {\it GraphX} pada {\it Spark} dan RDD.
		\item Melakukan eksperimen dari berbagai data graf dan membandingkan hasilnya dengan pemecahan graf menggunakan {\it GraphX} pada {\it Spark} .
		\item Menulis dokumen skripsi.
\end{enumerate}
	


\section{Sistematika Pembahasan}
\label{sec:sispem}
%Rencananya Bab 2 akan berisi petunjuk penggunaan template dan dasar-dasar \LaTeX.
%Mungkin bab 3,4,5 dapt diisi oleh ketiga jurusan, misalnya peraturan dasar skripsi atau pedoman penulisan, tentu jika berkenan.
%Bab 6 akan diisi dengan kesimpulan, bahwa membuat template ini ternyata sungguh menghabiskan banyak waktu.

%\dtext{10}

Bab 1 Pendahuluan\\
Bab 1 berisi latar belakang, rumusan masalah, tujuan, batasan masalah, metodologi penelitian dan sistematika pembahasan.\\


Bab 2 Dasar Teori\\
Bab 2 berisi \\


Bab 3 Analisis\\
Bab 3 berisi \\


Bab 4 Perancangan\\
Bab 4 berisi \\


Bab 5 Implementasi, Pengujian dan Eksperimen\\
Bab 5 berisi \\


Bab 6 Kesimpulan dan Saran\\
Bab 6 berisi kesimpulan dan saran.\\
